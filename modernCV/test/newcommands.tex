 %david Tocaven


%--------------------------------------------------------------------------------------------
%% 			Défintion de nouvelles commandes pour plus de souplesse dans moderncv.
%--------------------------------------------------------------------------------------------

% -----------------------------------------------------
% 				Paquets nécessaires
% -----------------------------------------------------

\usepackage{xspace} % gestion des espaces à la fin des commandes
\usepackage{soulutf8} % plus de libertée dans les soulignages
\usepackage[french]{babel}
\usepackage{xargs} % pour avoir plusieurs attribut qui ont des valeurs par defaut
% -----------------------------------------------------
% 				Nouvelles Commandes
% -----------------------------------------------------
	\definecolor{vert}{RGB}{16,150,24}



%---------- \partie{titre} : ma typo de section -------
% Entrée :
%#1=titre de la partie
\newcommand{\partie}[2][Large]{\setul{}{2pt} \par%
\noindent\hspace{1cm}%
\begin{minipage}[t]{.9\textwidth}
\textcolor{vert}{\begin{#1}{\ul{#2}}\end{#1}}%
\end{minipage}%
 \newline}
%------------------------------------------------------------------------------------------
%---------- \diplome{date}{nom du diplôme}{établissement}{lieu} : un bloc de diplôme-------
%------------------------------------------------------------------------------------------
% Entrées :
%#1=date
%#2=nom du diplôme
%#3=établissement
%#4=lieu
\newcommand{\diplome}[4]{\par%
\begin{minipage}[t]{2.3cm}%    la date
\begin{flushright}%
#1%
\end{flushright}%
\end{minipage}%
\hspace{2mm}%
%
\begin{minipage}[t]{\textwidth - 2.3cm -2mm}%   nom du diplôme, établissement et lieu
%
\begin{minipage}[t]{\textwidth}%  Non du dipôme
\textbf{#2,}\vspace{1mm}%
\end{minipage}%
\newline%
\begin{minipage}[t]{\textwidth}%    etablissement et lieu
\begin{flushleft}%
\textit{#3,\linebreak[2] #4.}%
\end{flushleft}%
\end{minipage}%
%
\end{minipage}%
\newline%
}
%-------------------------------------------------------------------------------------------------------------------------------------
%---------- \experience{date}{durée}{établissement}{lieu}{intitulé}{Points intéressants} : un bloc d'expérience professionnelle-------
%-------------------------------------------------------------------------------------------------------------------------------------
% Entrées :
%#1=date
%#2=durée
%#3=établissement
%#4=lieu
%#5=intitulé
%#6=Points intéressants (séparés par des --)
\newcommand{\experience}[6]{\par%
\begin{minipage}[t]{2.3cm}%    la date et la durée
\begin{flushleft}%
#1%date
\end{flushleft}%
\begin{center}%
\if #2\empty %
\else%
(#2)%durée
\fi%
\end{center}%
\end{minipage}%
\hspace{2mm}%
%
\begin{minipage}[t]{\textwidth - 2.3cm -2mm}% %
%
\begin{minipage}[t]{\textwidth}%
\textbf{#5, }\vspace{1mm}%intitulé
\textit{%
\if #3\empty %
\else%
#3%
%
\if #4\empty %
\else
,\linebreak[2] 
\fi
\fi%
\if #4\empty %
\else%
#4.%
\fi%
}%
\end{minipage}%
\newline%
\begin{minipage}[t]{\textwidth}%
\begin{flushleft}%
#6%
\end{flushleft}%
\end{minipage}%
%
\end{minipage}%
\newline%
}

%----------------------------------------------------------------------------------------------------------------------------------------
%---------- \tripleitemise[itemSymbole1]{text1}[itemSymbole2]{text2}[itemSymbole3]{text3} : une ligne avec 3 elements type itemise-------
%----------------------------------------------------------------------------------------------------------------------------------------
% Entrées :
%#1=itemSymbole1(=\textbullet)
%#2=text1
%#3=itemSymbole2(=\textbullet)
%#4=text2
%#5=itemSymbole3(=\textbullet)
%#6=text3
\newcommandx{\tripleitemise}[6][1=\textbullet,3=\textbullet,5=\textbullet]{\par%
% bloc 1
\begin{minipage}[t]{.33\textwidth}%
\if #2\empty% si text1 n'est pas vide
\else%
\noindent\begin{minipage}[t]{5mm}%
\noindent #1\hspace{3mm}%
\end{minipage}%
\noindent\begin{minipage}[t]{\textwidth - 5mm}%
\noindent\hspace{1mm} #2%
\end{minipage}%
\fi%
\end{minipage}\hfill%
% bloc 2
\noindent\begin{minipage}[t]{.33\textwidth}%
\if #4\empty% si text1 n'est pas vide
\else%
\noindent\begin{minipage}[t]{5mm}%
\noindent #3\hspace{3mm}%
\end{minipage}%
\noindent\begin{minipage}[t]{\textwidth - 5mm}%
\noindent\hspace{1mm} #4%
\end{minipage}%
\fi%
\end{minipage}\hfill%
% bloc 3
\begin{minipage}[t]{.33\textwidth}%
\if #6\empty% si text1 n'est pas vide
\else%
\noindent\begin{minipage}[t]{5mm}%
\noindent #5\hspace{3mm}%
\end{minipage}%
\noindent\begin{minipage}[t]{\textwidth - 5mm}%
\noindent\hspace{1mm} #6%
\end{minipage}%
\fi%
\end{minipage}%
\newline%
}
%----------------------------------------------------------------------------------------------------------------------------------------
%---------- \\tripleitemiseavecTitre{titre1}{text1}{titre2}{text2}{titre3}{text3}: trois colonnes avec un titre et une description par colonnes-------
%----------------------------------------------------------------------------------------------------------------------------------------
% Entrées :
%#1=titre1
%#2=text1
%#3=titre2
%#4=text2
%#5=titre3
%#6=text3
\newcommand{\tripleitemiseavecTitre}[6]{\par%
% bloc 1
\begin{minipage}[t]{.325\textwidth}%
\if #2\empty% si text1 n'est pas vide
\else%
\partie[large]{#1}%
{#2}%
\fi%
\end{minipage}\hfill %
% bloc 2
\noindent\begin{minipage}[t]{.325\textwidth}%
\if #4\empty% si text1 n'est pas vide
\else%
\partie[large]{#3}%
{#4}%
\fi%
\end{minipage}\hfill%
% bloc 3
\noindent\begin{minipage}[t]{.325\textwidth}%
\if #6\empty% si text1 n'est pas vide
\else%
\partie[large]{#5}%
{#6}%
\fi%
\end{minipage}\hfill%
\newline%
}
