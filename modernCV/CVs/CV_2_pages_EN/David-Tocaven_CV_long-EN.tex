%% start of file `template.tex'.
%% Copyright 2006-2015 Xavier Danaux (xdanaux@gmail.com).
%
% This work may be distributed and/or modified under the
% conditions of the LaTeX Project Public License version 1.3c,
% available at http://www.latex-project.org/lppl/.


\documentclass[11pt,a4paper,sans]{moderncv}        % possible options include font size ('10pt', '11pt' and '12pt'), paper size ('a4paper', 'letterpaper', 'a5paper', 'legalpaper', 'executivepaper' and 'landscape') and font family ('sans' and 'roman')

% moderncv themes
\moderncvstyle{classic}                             % style options are 'casual' (default), 'classic', 'banking', 'oldstyle' and 'fancy'
\moderncvcolor{green}                               % color options 'black', 'blue' (default), 'burgundy', 'green', 'grey', 'orange', 'purple' and 'red'
\usepackage[french]{babel}
%\renewcommand{\familydefault}{\sfdefault}         % to set the default font; use '\sfdefault' for the default sans serif font, '\rmdefault' for the default roman one, or any tex font name
%\nopagenumbers{}                                  % uncomment to suppress automatic page numbering for CVs longer than one page

% character encoding
\usepackage[utf8]{inputenc}                       % if you are not using xelatex ou lualatex, replace by the encoding you are using
%\usepackage{CJKutf8}                              % if you need to use CJK to typeset your resume in Chinese, Japanese or Korean
% adjust the page margins
\usepackage[scale=0.92]{geometry}
%\setlength{•}{•}\hintscolumnwidth}{3cm}                % if you want to change the width of the column with the dates
%\setlength{\makecvheadnamewidth}{10cm}            % for the 'classic' style, if you want to force the width allocated to your name and avoid line breaks. be careful though, the length is normally calculated to avoid any overlap with your personal info; use this at your own typographical risks...
\usepackage{graphicx}
\usepackage{float}
% personal data
\name{David}{\textbf{TOCAVEN}}
\title{\huge Étudiant master automatique}                               % optional, remove / comment the line if not wanted
\address{149 rue du Faubourg Bonnefoy}{31500 Toulouse}%{France}% optional, remove / comment the line if not wanted; the "postcode city" and "country" arguments can be omitted or provided empty
\phone[mobile]{(+33)6 45 52 25 72}                   % optional, remove / comment the line if not wanted; the optional "type" of the phone can be "mobile" (default), "fixed" or "fax"
%\phone[fixed]{+2~(345)~678~901}
%\phone[fax]{+3~(456)~789~012}
\email{david.tocaven@univ-tlse3.fr}                               % optional, remove / comment the line if not wanted
%\homepage{www.johndoe.com}                         % optional, remove / comment the line if not wanted
%\social[linkedin]{john.doe}                        % optional, remove / comment the line if not wanted
%\social[xing]{john\_doe}                           % optional, remove / comment the line if not wanted
%\social[twitter]{jdoe}                             % optional, remove / comment the line if not wanted
\social[github]{DavidTocaven}                              % optional, remove / comment the line if not wanted
%\social[gitlab]{jdoe}                              % optional, remove / comment the line if not wanted
%\social[skype]{jdoe}                               % optional, remove / comment the line if not wanted
\extrainfo{Français}                 % optional, remove / comment the line if not wanted
\photo[64pt][0.4pt]{photoID}                       % optional, remove / comment the line if not wanted; '64pt' is the height the picture must be resized to, 0.4pt is the thickness of the frame around it (put it to 0pt for no frame) and 'picture' is the name of the picture file
%\quote{Some quote}                                 % optional, remove / comment the line if not wanted

% bibliography adjustements (only useful if you make citations in your resume, or print a list of publications using BibTeX)
%   to show numerical labels in the bibliography (default is to show no labels)
%\makeatletter\renewcommand*{\bibliographyitemlabel}{\@biblabel{\arabic{enumiv}}}\makeatother
%\renewcommand*{\bibliographyitemlabel}{[\arabic{enumiv}]}
%   to redefine the bibliography heading string ("Publications")
%\renewcommand{\refname}{Articles}

% bibliography with mutiple entries
\usepackage{multibib}
%\usepackage{pifont}
%\usepackage{marvosym}
\usepackage{bclogo}
 %david Tocaven


%--------------------------------------------------------------------------------------------
%% 			Défintion de nouvelles commandes pour plus de souplesse dans moderncv.
%--------------------------------------------------------------------------------------------

% -----------------------------------------------------
% 				Paquets nécessaires
% -----------------------------------------------------

\usepackage{xspace} % gestion des espaces à la fin des commandes
\usepackage{soulutf8} % plus de libertée dans les soulignages
\usepackage[french]{babel}
\usepackage{xargs}% pour avoir plusieurs attribut qui ont des valeurs par defaut
\usepackage{fdsymbol}% des symbôles maths 
\usepackage{calc}
% -----------------------------------------------------
% 				Nouvelles Commandes
% -----------------------------------------------------
	\definecolor{vert}{RGB}{16,150,24}


%------------------------------------------------------
%---------- \partie[tailleTexte]{titre}[EpaisseurSoulignage] : ma typo de section -------
%------------------------------------------------------
% Entrées :
%#1=taille du texte (=Large)
%#2=titre de la partie
%#3=epaisseur du soulignage (=2pt)
\newcommandx{\partie}[3][1=Large,3=2pt]{\setul{}{#3}\par%
\noindent\hspace{8mm}\textcolor{vert}{\begin{#1}{$\smallblacktriangleright$}\end{#1}}\hspace{2mm}%
\begin{minipage}[t]{.9\textwidth}%
\textcolor{vert}{\begin{#1}{\ul{#2}}\end{#1}}%
\end{minipage}%
\newline%
}

%------------------------------------------------------
%---------- \souspartie[tailleTexte]{titre}[EpaisseurSoulignage] : ma typo de section -------
%------------------------------------------------------
% Entrées :
%#1=taille du texte (=Large)
%#2=titre de la partie
%#3=epaisseur du soulignage (=2pt)
\newcommandx{\souspartie}[3][1=Large,3=2pt]{\setul{}{#3}\par%
\noindent\hspace{3mm}\textcolor{vert}{\begin{#1}{$\smallblacktriangleright$}\end{#1}}\hspace{1mm}%
\begin{minipage}[t]{.9\textwidth}%
\noindent\textcolor{vert}{\begin{#1}{\ul{#2}}\end{#1}}%
\end{minipage}%
\par\noindent%
}

%------------------------------------------------------------------------------------------
%---------- \diplome{date}{nom du diplôme}{établissement}{lieu} : un bloc de diplôme-------
%------------------------------------------------------------------------------------------
% Entrées :
%#1=date
%#2=nom du diplôme
%#3=établissement
%#4=lieu


\newcommand{\diplome}[4]{\par%
\begin{minipage}[t]{2.3cm}%    la date
\begin{flushleft}%
#1%
\end{flushleft}%
\end{minipage}%
\hspace{2mm}%
%
\begin{minipage}[t]{.98\textwidth - 2.7cm}%   nom du diplôme, établissement et lieu
%
\begin{minipage}[t]{.98\textwidth}%  Non du dipôme
\textbf{#2,}\vspace{1mm}%
\end{minipage}%
\newline%
\begin{minipage}[t]{.98\textwidth}%    etablissement et lieu
\begin{flushleft}%
\textit{#3,\linebreak[2] #4.}%
\end{flushleft}%
\end{minipage}%
%
\end{minipage}%
\newline%
}
%-------------------------------------------------------------------------------------------------------------------------------------
%---------- \experience{date}{durée}{établissement}{lieu}{intitulé}{Points intéressants} : un bloc d'expérience professionnelle-------
%-------------------------------------------------------------------------------------------------------------------------------------
% Entrées :
%#1=date
%#2=durée
%#3=établissement
%#4=lieu
%#5=intitulé
%#6=Points intéressants (séparés par des --)
\newcommand{\experience}[6]{\par%
\begin{minipage}[t]{2.3cm}%    la date et la durée
#1\newline% date
\if #2\empty\else%
		(#2)% durée
\fi%
\end{minipage}%
\hspace{2mm}%
%
\begin{minipage}[t]{\textwidth - 2.3cm -2mm}% %
%
\begin{minipage}[t]{\textwidth}%
\textbf{#5, }\vspace{1mm}% intitulé
\textit{%
\if #3\empty\else #3%établissement
	\if #4\empty\else%
		,\linebreak[2]% 
	\fi%
\fi%
\if #4\empty\else%
	#4.%lieu
\fi%
}%
\end{minipage}%
\newline%
\begin{minipage}[t]{\textwidth}%
\begin{flushleft}%
#6%
\end{flushleft}%
\end{minipage}%
%
\end{minipage}%
\newline%
}

%----------------------------------------------------------------------------------------------------------------------------------------
%---------- \tripleitemise[itemSymbole1]{text1}[itemSymbole2]{text2}[itemSymbole3]{text3} : une ligne avec 3 elements type itemise-------
%----------------------------------------------------------------------------------------------------------------------------------------
% Entrées :
%#1=itemSymbole1(=\textbullet)
%#2=text1
%#3=itemSymbole2(=\textbullet)
%#4=text2
%#5=itemSymbole3(=\textbullet)
%#6=text3
\newcommandx{\tripleitemise}[6][1=\textbullet,3=\textbullet,5=\textbullet]{\par%
% bloc 1
\begin{minipage}[t]{.33\textwidth}%
\if #2\empty% si text1 n'est pas vide
\else%
\noindent\begin{minipage}[t]{5mm}%
\noindent #1\hspace{3mm}%
\end{minipage}%
\noindent\begin{minipage}[t]{\textwidth - 5mm}%
\noindent\hspace{1mm} #2%
\end{minipage}%
\fi%
\end{minipage}\hfill%
% bloc 2
\noindent\begin{minipage}[t]{.33\textwidth}%
\if #4\empty% si text1 n'est pas vide
\else%
\noindent\begin{minipage}[t]{5mm}%
\noindent #3\hspace{3mm}%
\end{minipage}%
\noindent\begin{minipage}[t]{\textwidth - 5mm}%
\noindent\hspace{1mm} #4%
\end{minipage}%
\fi%
\end{minipage}\hfill%
% bloc 3
\begin{minipage}[t]{.33\textwidth}%
\if #6\empty% si text1 n'est pas vide
\else%
\noindent\begin{minipage}[t]{5mm}%
\noindent #5\hspace{3mm}%
\end{minipage}%
\noindent\begin{minipage}[t]{\textwidth - 5mm}%
\noindent\hspace{1mm} #6%
\end{minipage}%
\fi%
\end{minipage}%
\par\noindent%
}
%----------------------------------------------------------------------------------------------------------------------------------------
%---------- \\tripleitemiseavecTitre{titre1}{text1}{titre2}{text2}{titre3}{text3}: trois colonnes avec un titre et une description par colonnes-------
%----------------------------------------------------------------------------------------------------------------------------------------
% Entrées :
%#1=titre1
%#2=text1
%#3=titre2
%#4=text2
%#5=titre3
%#6=text3
\newcommand{\tripleitemiseavecTitre}[6]{\par%
% titres 
\begin{minipage}[t]{\textwidth}%
%bloc 1
	\begin{minipage}[t]{.3\textwidth}%
	\if #1\empty\else%
	\noindent\souspartie[large]{#1}[1pt]%
	\fi%
	\end{minipage}\hfill %
% bloc 2
	\noindent\begin{minipage}[t]{.3\textwidth}%
	\if #3\empty\else%
	\noindent\souspartie[large]{#3}[1pt]%
	\fi%
	\end{minipage}\hfill%
% bloc 3
	\noindent\begin{minipage}[t]{.32\textwidth}%
	\if #5\empty\else%
	\noindent\souspartie[large]{#5}[1pt]%
	\fi%
	\end{minipage}\hfill%
\end{minipage}%
\newline%
%\vspace{3mm}
% corps de textes
\begin{minipage}[t]{\textwidth}%
\hfill %
% bloc 1

	\begin{minipage}[t]{.3\textwidth}%
	{#2}%
	\end{minipage}%
\hfill %\vline \hfill %
% bloc 2
	\begin{minipage}[t]{.3\textwidth}%
	{#4}%
	\end{minipage}%
\hfill %\vline \hfill %
% bloc 3
	\begin{minipage}[t]{.32\textwidth}%
	{#6}%
	\end{minipage}\hfill%
\end{minipage}
\vspace{3mm}\par\noindent%
}



%-----------------------------------------------------------------------------
%---------- \miniinclude{img} : inclue une image de la taille du texte -------
%-----------------------------------------------------------------------------
% Entrées :
%#1=image a inclure
\newcommand*{\miniinclude}[1]{%
  \raisebox{-.3\baselineskip}{%
    \includegraphics[scale=•]{•}
      height=\baselineskip,
      width=\baselineskip,
      keepaspectratio,
    ]{#1}%
  }%
}
%\newcites{book,misc}{{Books},{Others}}
\usepackage{enumitem}
\newcommand{\myitem}{\textbullet}

	\definecolor{darkred}{RGB}{204,0,0}


%----------------------------------------------------------------------------------
%            content
%----------------------------------------------------------------------------------
\begin{document}

\makecvtitle
%\setlength{\maincolumnwidth}{\linewidth-\leftskip-\rightskip-\separatorcolumnwidth-\hintscolumnwidth}
\partie{Diplômes}%
\vspace{2mm}
\diplome{2015 à ce jour}{Master EEA-Ingénierie des Systèmes Temps-Réels}{\textcolor{darkred}{Université Toulouse III -- Paul Sabatier}}{\textcolor{darkred}{Toulouse}}% 
\vspace{2mm}
\diplome{2013--2015}{Licence Électronique, Électrotechnique et Automatique}{\textcolor{darkred}{Université Toulouse III -- Paul Sabatier}}{\textcolor{darkred}{Toulouse}}%
\vspace{2mm}
\diplome{2010--2013}{Baccalauréat Série Scientifique}{\textcolor{darkred}{Lycée La Borde Basse}}{\textcolor{darkred}{Castres}}

\vspace{4mm}
%
% -----------------------
\partie{Expériences}
% -----------------------
%
\vspace{2mm}

%\experience{date}{durée}{établissement}{lieu}{intitulé}{Points intéressants}
%
\experience{2016--2017}{4 semaines}{{\textcolor{darkred}{\\LAAS-CNRS}}}{{\textcolor{darkred}{Toulouse}}}{Mission : Modélisation de S.E.D (Système à Événements Discrets)}{ \vspace{1mm} \begin{itemize}[label=\myitem]
\item Modèle DEVS,
\item Modélisation temps~discrets, 
\item Systèmes à événements discrets.
\end{itemize}}

\vspace{4mm}

\experience{2016--2017}{6 mois}{{\textcolor{darkred}{\\Université Toulouse III -- Paul Sabatier}}}{{\textcolor{darkred}{Toulouse}}}{Modélisation, Analyse et Simulation des S.E.D non déterministes}{ \vspace{1mm} \begin{itemize}[label=\myitem]
\item Méthode scientifique et gestion~de~projet,
\item Formalisme automate,
\item Matlab.
\end{itemize}}

\vspace{4mm}

\experience{2016--2017}{5 semaines}{{\textcolor{darkred}{\\LAPLACE}}}{{\textcolor{darkred}{Toulouse}}}{Mission : Visualisation et quantification des échanges thermiques par \\convection sur un dissipateur pour sources d’éclairages à LED}{ \vspace{1mm} \begin{itemize}[label=\myitem]
\item Optique et traitement~d’image, 
\item Matlab et \LaTeX, 
\item Découverte du monde de la recherche.
\end{itemize}}

\vspace{4mm}

\experience{2016--2017}{1 an}{}{\textcolor{darkred}{\\Toulouse}}{Président de l'association étudiante EEA Toulouse}{ \vspace{1mm} \begin{itemize}[label=\myitem]
\item Conduite de réunion,
\item Organisation d'événements,
\item Communication.
\end{itemize}}

\vspace{4mm}

\experience{2016 à aujourd'hui}{}{}{{\textcolor{darkred}{\\Toulouse}}}{Cours particulier}{ \vspace{1mm} \begin{itemize}[label=\myitem]
\item Pédagogie, 
\item Mathématique et automatique, 
\item Visualisation.
\end{itemize}}
\vspace{4mm}
% -----------------------
\partie{Compétences}
% -----------------------
\vspace{2mm}
%\begin{cvcolumns}
%\tripleitemiseavecTitre{coucou}{{\begin{itemize}\item 1 \item 2 \end{itemize}}}{ça}{3 4}{va}{5 6}

%\noindent\hspace{3cm}\souspartie[large]{Automatique}[1pt]
\tripleitemiseavecTitre{Automatique -- Temps continu ou échantillonné}{%
{\begin{itemize}[label=\myitem]%
\item \textbf{Modélisation :}\\ Fréquentielle, d'espace d'état, linéaire et non linéaire, linéaire multivariable, d'observateurs, incertaine, de système à retard.%
\item \textbf{Analyse :} \\ Fréquentielle, Théorie de Lyapunov, de performance, de système linéaire et non linéaire, de système incertain, de robustesse, de stabilité sur systèmes à retard.%
\item \textbf{Synthèse de commande :}  \\ PID, multivariable, robuste, par retour d'état, sur système a retard.
\end{itemize}}}%
%
{Automatique -- Systèmes à événements discret}{{\begin{itemize}[label=\myitem]%
		\item Automates,%
		\item {  Réseaux de Petri (normaux, stochastiques/temporisés),}%
		\item {  Algèbre $(max,+)$,}%
		\item {  Modèles DEVS,}%
		\item {  Synthèse de commande,}%
		\item {  Analyse, simulation, implémentation,}%
		\item {  Diagnostiqueur et Contrôlabilité,}%
		\item {  Langage,}%
		\item {  Commande supervisée,}%
%		\item {  Mise en \oe uvre logicielle/matérielle}%
		\end{itemize}}}%
%
{Mise en \oe uvre}{{%
\begin{itemize}[label=\myitem]%
\item \textbf{Informatique :}\\%
{ Modélisation système (UML, UML2, SysML, systèmes embarqués), orienté objet, parallèle (exclusion mutuelle, synchronisation, thread , multitâches.).}%
%
\item \textbf{Informatique industrielle :}\\%
 { DSP \textit{notions}, Micro-contrôleurs \textit{bases},} %
 %
\item \textbf{Temps Réel :}\\%
{  Norme OSEK/VDX, Ordonnancement, RTOS, Vérification d’exigence, Réactivité.}%
%
\item \textbf{Réseaux :}\\%
{  Internet \textit{bases}, Network Calculus, CAN, AFDX, Réseaux temps réel.}%
\end{itemize}%
}}%

\vspace{4mm}

\tripleitemiseavecTitre%
{Logiciels}%
{{%
	\begin{itemize}[label=\myitem]
		\item \textbf{Spécifiques automatique :}\\%
\textbf{Matlab}: Simulink, POO, GUI, RTW, 
		\item \textbf{Spécifique Événements discrets :}\\%
\textbf{DESUMA}, \textbf{Tina}, \textbf{ProDevs}, 
		\item \textbf{Spécifique Informatique :}\\%
 \textbf{Eclipse}, \textbf{Git}, \textbf{Doxygen}, \textbf{MagicDraw}, \textbf{Modelio}.
		\item \textbf{Bureautique :} \\%
\textbf{\TeX maker}, \textbf{Suites Microsoft office}, \textbf{Libre Office}, \textbf{Gimp}, \textbf{Inkscape},.
	\end{itemize}
}}%
{Langages}%
{%
	\begin{itemize}[label=\myitem]
		\item \textbf{Matlab} \textit{bonnes connaissances}, 
		\item \textbf{\LaTeX}~\textit{bonnes connaissances}, 
		\item \textbf{C}~\textit{bonnes connaissances},
		\item \textbf{Assembleur} \textit{notions}, 
		\item \textbf{VHDL} \textit{bonnes bases}, 
		\item \textbf{ST}~et~\textbf{IL-LIST}~\textit{notions}, 
		\item \textbf{Arduino}~\textit{bonnes bases}, 
		\item \textbf{Java}~\textit{bases}, 
		\item \textbf{C++}~\textit{notions}.%
	\end{itemize}
}%
{Langue et communication}%
{{%
	\begin{itemize}[label=\myitem]%
	\item \textbf{Langue: } \\ Anglais B2%
	\item \textbf{Communication: } \\ Orale/écrite/visuelle -- \\Français/Anglais. %
	\end{itemize}%
	\vspace{2mm}
	\noindent\souspartie[large]{Gestion/management de projet}[1pt]\vspace{0mm}
\begin{itemize}[label=\myitem]%
\item Conduite de réunion,
\item Organisation et suivi d'un projet,
\item Cycle en V,
\item Méthodes agiles.
\end{itemize}
	\textbf{(Gantt, WBS, RACI)}%
}}
\vspace{4mm}
% -----------------------
\partie{Investissements personnels}
\vspace{2mm}

% -----------------------
%\tripleitemise[itemSymbole1]{text1}[itemSymbole2]{text2}[itemSymbole3]{text3}
\tripleitemise[\bcbook]{Veille scientifique}[\bcfleur]{Voyages}[\bcvelo]{Bricolage (remorque vélo, électronique, …)}
\tripleitemise{}{Bénévole IFAC 2017}{}
%
\end{document}


%% end of file `template.tex'.
