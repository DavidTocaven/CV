 %david Tocaven


%--------------------------------------------------------------------------------------------
%% 			Défintion de nouvelles commandes pour plus de souplesse dans moderncv.
%--------------------------------------------------------------------------------------------

% -----------------------------------------------------
% 				Paquets nécessaires
% -----------------------------------------------------

\usepackage{xspace} % gestion des espaces à la fin des commandes
\usepackage{soulutf8} % plus de libertée dans les soulignages
\usepackage[french]{babel}
\usepackage{xargs}% pour avoir plusieurs attribut qui ont des valeurs par defaut
\usepackage{fdsymbol}% des symbôles maths 
\usepackage{calc}
% -----------------------------------------------------
% 				Nouvelles Commandes
% -----------------------------------------------------
	\definecolor{vert}{RGB}{16,150,24}


%------------------------------------------------------
%---------- \partie[tailleTexte]{titre}[EpaisseurSoulignage] : ma typo de section -------
%------------------------------------------------------
% Entrées :
%#1=taille du texte (=Large)
%#2=titre de la partie
%#3=epaisseur du soulignage (=2pt)
\newcommandx{\partie}[3][1=Large,3=2pt]{\setul{}{#3}\par%
\noindent\hspace{8mm}\textcolor{vert}{\begin{#1}{$\smallblacktriangleright$}\end{#1}}\hspace{2mm}%
\begin{minipage}[t]{.9\textwidth}%
\textcolor{vert}{\begin{#1}{\ul{#2}}\end{#1}}%
\end{minipage}%
\newline%
}

%------------------------------------------------------
%---------- \souspartie[tailleTexte]{titre}[EpaisseurSoulignage] : ma typo de section -------
%------------------------------------------------------
% Entrées :
%#1=taille du texte (=Large)
%#2=titre de la partie
%#3=epaisseur du soulignage (=2pt)
\newcommandx{\souspartie}[3][1=Large,3=2pt]{\setul{}{#3}\par%
\noindent\hspace{3mm}\textcolor{vert}{\begin{#1}{$\smallblacktriangleright$}\end{#1}}\hspace{1mm}%
\begin{minipage}[t]{.9\textwidth}%
\noindent\textcolor{vert}{\begin{#1}{\ul{#2}}\end{#1}}%
\end{minipage}%
\par\noindent%
}

%------------------------------------------------------------------------------------------
%---------- \diplome{date}{nom du diplôme}{établissement}{lieu} : un bloc de diplôme-------
%------------------------------------------------------------------------------------------
% Entrées :
%#1=date
%#2=nom du diplôme
%#3=établissement
%#4=lieu


\newcommand{\diplome}[4]{\par%
\begin{minipage}[t]{2.3cm}%    la date
\begin{flushleft}%
#1%
\end{flushleft}%
\end{minipage}%
\hspace{2mm}%
%
\begin{minipage}[t]{.98\textwidth - 2.7cm}%   nom du diplôme, établissement et lieu
%
\begin{minipage}[t]{.98\textwidth}%  Non du dipôme
\textbf{#2,}\vspace{1mm}%
\end{minipage}%
\newline%
\begin{minipage}[t]{.98\textwidth}%    etablissement et lieu
\begin{flushleft}%
\textit{#3,\linebreak[2] #4.}%
\end{flushleft}%
\end{minipage}%
%
\end{minipage}%
\newline%
}
%-------------------------------------------------------------------------------------------------------------------------------------
%---------- \experience{date}{durée}{établissement}{lieu}{intitulé}{Points intéressants} : un bloc d'expérience professionnelle-------
%-------------------------------------------------------------------------------------------------------------------------------------
% Entrées :
%#1=date
%#2=durée
%#3=établissement
%#4=lieu
%#5=intitulé
%#6=Points intéressants (séparés par des --)
\newcommand{\experience}[6]{\par%
\begin{minipage}[t]{2.3cm}%    la date et la durée
#1\newline% date
\if #2\empty\else%
		(#2)% durée
\fi%
\end{minipage}%
\hspace{2mm}%
%
\begin{minipage}[t]{\textwidth - 2.3cm -2mm}% %
%
\begin{minipage}[t]{\textwidth}%
\textbf{#5, }\vspace{1mm}% intitulé
\textit{%
\if #3\empty\else #3%établissement
	\if #4\empty\else%
		,\linebreak[2]% 
	\fi%
\fi%
\if #4\empty\else%
	#4.%lieu
\fi%
}%
\end{minipage}%
\newline%
\begin{minipage}[t]{\textwidth}%
\begin{flushleft}%
#6%
\end{flushleft}%
\end{minipage}%
%
\end{minipage}%
\newline%
}

%----------------------------------------------------------------------------------------------------------------------------------------
%---------- \tripleitemise[itemSymbole1]{text1}[itemSymbole2]{text2}[itemSymbole3]{text3} : une ligne avec 3 elements type itemise-------
%----------------------------------------------------------------------------------------------------------------------------------------
% Entrées :
%#1=itemSymbole1(=\textbullet)
%#2=text1
%#3=itemSymbole2(=\textbullet)
%#4=text2
%#5=itemSymbole3(=\textbullet)
%#6=text3
\newcommandx{\tripleitemise}[6][1=\textbullet,3=\textbullet,5=\textbullet]{\par%
% bloc 1
\begin{minipage}[t]{.33\textwidth}%
\if #2\empty% si text1 n'est pas vide
\else%
\noindent\begin{minipage}[t]{5mm}%
\noindent #1\hspace{3mm}%
\end{minipage}%
\noindent\begin{minipage}[t]{\textwidth - 5mm}%
\noindent\hspace{1mm} #2%
\end{minipage}%
\fi%
\end{minipage}\hfill%
% bloc 2
\noindent\begin{minipage}[t]{.33\textwidth}%
\if #4\empty% si text1 n'est pas vide
\else%
\noindent\begin{minipage}[t]{5mm}%
\noindent #3\hspace{3mm}%
\end{minipage}%
\noindent\begin{minipage}[t]{\textwidth - 5mm}%
\noindent\hspace{1mm} #4%
\end{minipage}%
\fi%
\end{minipage}\hfill%
% bloc 3
\begin{minipage}[t]{.33\textwidth}%
\if #6\empty% si text1 n'est pas vide
\else%
\noindent\begin{minipage}[t]{5mm}%
\noindent #5\hspace{3mm}%
\end{minipage}%
\noindent\begin{minipage}[t]{\textwidth - 5mm}%
\noindent\hspace{1mm} #6%
\end{minipage}%
\fi%
\end{minipage}%
\par\noindent%
}
%----------------------------------------------------------------------------------------------------------------------------------------
%---------- \\tripleitemiseavecTitre{titre1}{text1}{titre2}{text2}{titre3}{text3}: trois colonnes avec un titre et une description par colonnes-------
%----------------------------------------------------------------------------------------------------------------------------------------
% Entrées :
%#1=titre1
%#2=text1
%#3=titre2
%#4=text2
%#5=titre3
%#6=text3
\newcommand{\tripleitemiseavecTitre}[6]{\par%
% titres 
\begin{minipage}[t]{\textwidth}%
%bloc 1
	\begin{minipage}[t]{.3\textwidth}%
	\if #1\empty\else%
	\noindent\souspartie[large]{#1}[1pt]%
	\fi%
	\end{minipage}\hfill %
% bloc 2
	\noindent\begin{minipage}[t]{.3\textwidth}%
	\if #3\empty\else%
	\noindent\souspartie[large]{#3}[1pt]%
	\fi%
	\end{minipage}\hfill%
% bloc 3
	\noindent\begin{minipage}[t]{.32\textwidth}%
	\if #5\empty\else%
	\noindent\souspartie[large]{#5}[1pt]%
	\fi%
	\end{minipage}\hfill%
\end{minipage}%
\newline%
%\vspace{3mm}
% corps de textes
\begin{minipage}[t]{\textwidth}%
\hfill %
% bloc 1

	\begin{minipage}[t]{.3\textwidth}%
	{#2}%
	\end{minipage}%
\hfill %\vline \hfill %
% bloc 2
	\begin{minipage}[t]{.3\textwidth}%
	{#4}%
	\end{minipage}%
\hfill %\vline \hfill %
% bloc 3
	\begin{minipage}[t]{.32\textwidth}%
	{#6}%
	\end{minipage}\hfill%
\end{minipage}
\vspace{3mm}\par\noindent%
}



%-----------------------------------------------------------------------------
%---------- \miniinclude{img} : inclue une image de la taille du texte -------
%-----------------------------------------------------------------------------
% Entrées :
%#1=image a inclure
\newcommand*{\miniinclude}[1]{%
  \raisebox{-.3\baselineskip}{%
    \includegraphics[scale=•]{•}
      height=\baselineskip,
      width=\baselineskip,
      keepaspectratio,
    ]{#1}%
  }%
}