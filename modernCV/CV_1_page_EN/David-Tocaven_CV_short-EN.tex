%% start of file `template.tex'.
%% Copyright 2006-2015 Xavier Danaux (xdanaux@gmail.com).
%
% This work may be distributed and/or modified under the
% conditions of the LaTeX Project Public License version 1.3c,
% available at http://www.latex-project.org/lppl/.


\documentclass[11pt,a4paper,sans]{moderncv}        % possible options include font size ('10pt', '11pt' and '12pt'), paper size ('a4paper', 'letterpaper', 'a5paper', 'legalpaper', 'executivepaper' and 'landscape') and font family ('sans' and 'roman')

% moderncv themes
\moderncvstyle{classic}                             % style options are 'casual' (default), 'classic', 'banking', 'oldstyle' and 'fancy'
\moderncvcolor{green}                               % color options 'black', 'blue' (default), 'burgundy', 'green', 'grey', 'orange', 'purple' and 'red'
\usepackage[french]{babel}
%\renewcommand{\familydefault}{\sfdefault}         % to set the default font; use '\sfdefault' for the default sans serif font, '\rmdefault' for the default roman one, or any tex font name
%\nopagenumbers{}                                  % uncomment to suppress automatic page numbering for CVs longer than one page

% character encoding
\usepackage[utf8]{inputenc}                       % if you are not using xelatex ou lualatex, replace by the encoding you are using
%\usepackage{CJKutf8}                              % if you need to use CJK to typeset your resume in Chinese, Japanese or Korean
% adjust the page margins
\usepackage[scale=0.92,margin=8mm]{geometry}
%\setlength{•}{•}\hintscolumnwidth}{3cm}                % if you want to change the width of the column with the dates
%\setlength{\makecvheadnamewidth}{10cm}            % for the 'classic' style, if you want to force the width allocated to your name and avoid line breaks. be careful though, the length is normally calculated to avoid any overlap with your personal info; use this at your own typographical risks...

% personal data
\name{David}{\textbf{TOCAVEN}}
\title{\huge Automatic master's degree student} % fr : Étudiant master automatique                               % optional, remove / comment the line if not wanted
\address{149 rue du Faubourg Bonnefoy}{31500 Toulouse}{France}% optional, remove / comment the line if not wanted; the "postcode city" and "country" arguments can be omitted or provided empty
\phone[mobile]{(+33)6 45 52 25 72}                   % optional, remove / comment the line if not wanted; the optional "type" of the phone can be "mobile" (default), "fixed" or "fax"
%\phone[fixed]{+2~(345)~678~901}
%\phone[fax]{+3~(456)~789~012}
\email{david.tocaven@univ-tlse3.fr}                               % optional, remove / comment the line if not wanted
%\homepage{www.johndoe.com}                         % optional, remove / comment the line if not wanted
%\social[linkedin]{john.doe}                        % optional, remove / comment the line if not wanted
%\social[xing]{john\_doe}                           % optional, remove / comment the line if not wanted
%\social[twitter]{jdoe}                             % optional, remove / comment the line if not wanted
\social[github]{DavidTocaven}                              % optional, remove / comment the line if not wanted
%\social[gitlab]{jdoe}                              % optional, remove / comment the line if not wanted
%\social[skype]{jdoe}                               % optional, remove / comment the line if not wanted
\extrainfo{French\\Driver's licence}                 % optional, remove / comment the line if not wanted
\photo[64pt][0.4pt]{photoID}                       % optional, remove / comment the line if not wanted; '64pt' is the height the picture must be resized to, 0.4pt is the thickness of the frame around it (put it to 0pt for no frame) and 'picture' is the name of the picture file
%\quote{Some quote}                                 % optional, remove / comment the line if not wanted

% bibliography adjustements (only useful if you make citations in your resume, or print a list of publications using BibTeX)
%   to show numerical labels in the bibliography (default is to show no labels)
%\makeatletter\renewcommand*{\bibliographyitemlabel}{\@biblabel{\arabic{enumiv}}}\makeatother
%\renewcommand*{\bibliographyitemlabel}{[\arabic{enumiv}]}
%   to redefine the bibliography heading string ("Publications")
%\renewcommand{\refname}{Articles}

% bibliography with mutiple entries
%\usepackage{multibib}
%\usepackage{pifont}
%\usepackage{marvosym}
\usepackage{bclogo}
 %david Tocaven


%--------------------------------------------------------------------------------------------
%% 			Défintion de nouvelles commandes pour plus de souplesse dans moderncv.
%--------------------------------------------------------------------------------------------

% -----------------------------------------------------
% 				Paquets nécessaires
% -----------------------------------------------------

\usepackage{xspace} % gestion des espaces à la fin des commandes
\usepackage{soulutf8} % plus de libertée dans les soulignages
\usepackage[french]{babel}
\usepackage{xargs}% pour avoir plusieurs attribut qui ont des valeurs par defaut
\usepackage{fdsymbol}% des symbôles maths 
\usepackage{calc}
% -----------------------------------------------------
% 				Nouvelles Commandes
% -----------------------------------------------------
	\definecolor{vert}{RGB}{16,150,24}


%------------------------------------------------------
%---------- \partie[tailleTexte]{titre}[EpaisseurSoulignage] : ma typo de section -------
%------------------------------------------------------
% Entrées :
%#1=taille du texte (=Large)
%#2=titre de la partie
%#3=epaisseur du soulignage (=2pt)
\newcommandx{\partie}[3][1=Large,3=2pt]{\setul{}{#3}\par%
\noindent\hspace{8mm}\textcolor{vert}{\begin{#1}{$\smallblacktriangleright$}\end{#1}}\hspace{2mm}%
\begin{minipage}[t]{.9\textwidth}%
\textcolor{vert}{\begin{#1}{\ul{#2}}\end{#1}}%
\end{minipage}%
\newline%
}

%------------------------------------------------------
%---------- \souspartie[tailleTexte]{titre}[EpaisseurSoulignage] : ma typo de section -------
%------------------------------------------------------
% Entrées :
%#1=taille du texte (=Large)
%#2=titre de la partie
%#3=epaisseur du soulignage (=2pt)
\newcommandx{\souspartie}[3][1=Large,3=2pt]{\setul{}{#3}\par%
\noindent\hspace{3mm}\textcolor{vert}{\begin{#1}{$\smallblacktriangleright$}\end{#1}}\hspace{1mm}%
\begin{minipage}[t]{.9\textwidth}%
\noindent\textcolor{vert}{\begin{#1}{\ul{#2}}\end{#1}}%
\end{minipage}%
\par\noindent%
}

%------------------------------------------------------------------------------------------
%---------- \diplome{date}{nom du diplôme}{établissement}{lieu} : un bloc de diplôme-------
%------------------------------------------------------------------------------------------
% Entrées :
%#1=date
%#2=nom du diplôme
%#3=établissement
%#4=lieu


\newcommand{\diplome}[4]{\par%
\begin{minipage}[t]{2.3cm}%    la date
\begin{flushleft}%
#1%
\end{flushleft}%
\end{minipage}%
\hspace{2mm}%
%
\begin{minipage}[t]{.98\textwidth - 2.7cm}%   nom du diplôme, établissement et lieu
%
\begin{minipage}[t]{.98\textwidth}%  Non du dipôme
\textbf{#2,}\vspace{1mm}%
\end{minipage}%
\newline%
\begin{minipage}[t]{.98\textwidth}%    etablissement et lieu
\begin{flushleft}%
\textit{#3,\linebreak[2] #4.}%
\end{flushleft}%
\end{minipage}%
%
\end{minipage}%
\newline%
}
%-------------------------------------------------------------------------------------------------------------------------------------
%---------- \experience{date}{durée}{établissement}{lieu}{intitulé}{Points intéressants} : un bloc d'expérience professionnelle-------
%-------------------------------------------------------------------------------------------------------------------------------------
% Entrées :
%#1=date
%#2=durée
%#3=établissement
%#4=lieu
%#5=intitulé
%#6=Points intéressants (séparés par des --)
\newcommand{\experience}[6]{\par%
\begin{minipage}[t]{2.3cm}%    la date et la durée
#1\newline% date
\if #2\empty\else%
		(#2)% durée
\fi%
\end{minipage}%
\hspace{2mm}%
%
\begin{minipage}[t]{\textwidth - 2.3cm -2mm}% %
%
\begin{minipage}[t]{\textwidth}%
\textbf{#5, }\vspace{1mm}% intitulé
\textit{%
\if #3\empty\else #3%établissement
	\if #4\empty\else%
		,\linebreak[2]% 
	\fi%
\fi%
\if #4\empty\else%
	#4.%lieu
\fi%
}%
\end{minipage}%
\newline%
\begin{minipage}[t]{\textwidth}%
\begin{flushleft}%
#6%
\end{flushleft}%
\end{minipage}%
%
\end{minipage}%
\newline%
}

%----------------------------------------------------------------------------------------------------------------------------------------
%---------- \tripleitemise[itemSymbole1]{text1}[itemSymbole2]{text2}[itemSymbole3]{text3} : une ligne avec 3 elements type itemise-------
%----------------------------------------------------------------------------------------------------------------------------------------
% Entrées :
%#1=itemSymbole1(=\textbullet)
%#2=text1
%#3=itemSymbole2(=\textbullet)
%#4=text2
%#5=itemSymbole3(=\textbullet)
%#6=text3
\newcommandx{\tripleitemise}[6][1=\textbullet,3=\textbullet,5=\textbullet]{\par%
% bloc 1
\begin{minipage}[t]{.33\textwidth}%
\if #2\empty% si text1 n'est pas vide
\else%
\noindent\begin{minipage}[t]{5mm}%
\noindent #1\hspace{3mm}%
\end{minipage}%
\noindent\begin{minipage}[t]{\textwidth - 5mm}%
\noindent\hspace{1mm} #2%
\end{minipage}%
\fi%
\end{minipage}\hfill%
% bloc 2
\noindent\begin{minipage}[t]{.33\textwidth}%
\if #4\empty% si text1 n'est pas vide
\else%
\noindent\begin{minipage}[t]{5mm}%
\noindent #3\hspace{3mm}%
\end{minipage}%
\noindent\begin{minipage}[t]{\textwidth - 5mm}%
\noindent\hspace{1mm} #4%
\end{minipage}%
\fi%
\end{minipage}\hfill%
% bloc 3
\begin{minipage}[t]{.33\textwidth}%
\if #6\empty% si text1 n'est pas vide
\else%
\noindent\begin{minipage}[t]{5mm}%
\noindent #5\hspace{3mm}%
\end{minipage}%
\noindent\begin{minipage}[t]{\textwidth - 5mm}%
\noindent\hspace{1mm} #6%
\end{minipage}%
\fi%
\end{minipage}%
\par\noindent%
}
%----------------------------------------------------------------------------------------------------------------------------------------
%---------- \\tripleitemiseavecTitre{titre1}{text1}{titre2}{text2}{titre3}{text3}: trois colonnes avec un titre et une description par colonnes-------
%----------------------------------------------------------------------------------------------------------------------------------------
% Entrées :
%#1=titre1
%#2=text1
%#3=titre2
%#4=text2
%#5=titre3
%#6=text3
\newcommand{\tripleitemiseavecTitre}[6]{\par%
% titres 
\begin{minipage}[t]{\textwidth}%
%bloc 1
	\begin{minipage}[t]{.325\textwidth}%
	\if #1\empty\else%
	\noindent\souspartie[large]{#1}[1pt]%
	\fi%
	\end{minipage}\hfill %
% bloc 2
	\noindent\begin{minipage}[t]{.325\textwidth}%
	\if #3\empty\else%
	\noindent\souspartie[large]{#3}[1pt]%
	\fi%
	\end{minipage}\hfill%
% bloc 3
	\noindent\begin{minipage}[t]{.325\textwidth}%
	\if #5\empty\else%
	\noindent\souspartie[large]{#5}[1pt]%
	\fi%
	\end{minipage}\hfill%
\end{minipage}%
\newline%
%\vspace{3mm}
% corps de textes
\begin{minipage}[t]{\textwidth}%
\hfill %
% bloc 1

	\begin{minipage}[t]{.322\textwidth}%
	{#2}%
	\end{minipage}%
\hfill %\vline \hfill %
% bloc 2
	\begin{minipage}[t]{.322\textwidth}%
	{#4}%
	\end{minipage}%
\hfill %\vline \hfill %
% bloc 3
	\begin{minipage}[t]{.322\textwidth}%
	{#6}%
	\end{minipage}\hfill%
\end{minipage}
\vspace{3mm}\par\noindent%
}



%-----------------------------------------------------------------------------
%---------- \miniinclude{img} : inclue une image de la taille du texte -------
%-----------------------------------------------------------------------------
% Entrées :
%#1=image a inclure
\newcommand*{\miniinclude}[1]{%
  \raisebox{-.3\baselineskip}{%
    \includegraphics[scale=•]{•}
      height=\baselineskip,
      width=\baselineskip,
      keepaspectratio,
    ]{#1}%
  }%
}
%\newcites{book,misc}{{Books},{Others}}
\usepackage{enumitem}
\newcommand{\myitem}{\textbullet}

	\definecolor{darkred}{RGB}{204,0,0}


%----------------------------------------------------------------------------------
%            content
%----------------------------------------------------------------------------------
\begin{document}
\begin{samepage}
%\begin{CJK*}{UTF8}{gbsn}                          % to typeset your resume in Chinese using CJK
%-----       resume       ---------------------------------------------------------
\makecvtitle
%\setlength{\maincolumnwidth}{\linewidth-\leftskip-\rightskip-\separatorcolumnwidth-\hintscolumnwidth}
\vspace{-1.3cm}%
\noindent\begin{minipage}[t]{.44\textwidth}%
% -----------------------
\vspace*{-3mm}%
\partie{Education}% fr : Diplômes 
% -----------------------
\diplome{2015 -- 2017}{Real-time systems engineering - EEA master's degree}%Master EEA-Ingénierie des Systèmes Temps-Réels
{\textcolor{darkred}{Paul Sabatier University}}%Université Toulouse III -- Paul Sabatier
{\textcolor{darkred}{Toulouse}}%Toulouse
%
\diplome{2013 -- 2015}{Electronic, Electronic engineering and automatic Bachelor's degree}%Licence Électronique, Électrotechnique et Automatique
{\textcolor{darkred}{Paul Sabatier University}}%Université Toulouse III -- Paul Sabatier
{\textcolor{darkred}{Toulouse}}%Toulouse
%
\diplome{2010 -- 2013}{Scientific stream Baccalaureate \\ \textmd{ \small (equivalent to Hight School diploma)}}%Baccalauréat Série Scientifique
{\textcolor{darkred}{La Borde Basse Hight School}}%Lycée La Borde Basse
{\textcolor{darkred}{Castres}}%Castres
\end{minipage}\hfill%
\noindent\begin{minipage}[t]{.51\textwidth}
% -----------------------
\partie{Work Experience}
% -----------------------
%\experience{date}{durée}{établissement}{lieu}{intitulé}{Points intéressants}
\experience{Apr. to Aug. 2018}{5 months}{LAAS-CNRS}{Toulouse}%
{Research internship}%Stage de recherche 
{Active diagnostic, hybrid system, observer, parity space}%diagnostic actif, système hybride, RRA, observer, parity space
\experience{2016--2017}{4 weeks}{LAAS-CNRS}{Toulouse}%
{Research internship}%Stage
{DEVS model, discrete time, discrete events, modelling}%Modèle DEVS, temps~discrets, événements~discrets, modélisation.

\experience{2016 -- 2017}{6 months}%
{Paul Sabatier University}%Université Toulouse III -- Paul Sabatier
{Toulouse}%Toulouse
{Master project}%Projet master
{Scientific method, automaton, project management, Matlab}%Méthode scientifique, automates, gestion~de~projet, Matlab

\experience{2016 -- 2017}{5 weeks}{LAPLACE}{Toulouse}%
{Research internship}%Stage
{Optic, digital image processing, thermal science, Matlab, \LaTeX , Discovering the research world}%Optique, traitement~d’image, thermique, Matlab, \LaTeX, découverte du monde de la recherche.

\experience{2016\
$\!\text{to}\!$ present}{}{}{Toulouse}%
{Private lesson}%Cours particulier
{Mathematics and automatic, \\
Teaching skills and mathematical visualization}%Pédagogie, mathématique, automatique, visualisation
\end{minipage}% 
\vspace*{-5mm}% -----------------------
\partie{Skills}%
% -----------------------
%\begin{cvcolumns}
%\tripleitemiseavecTitre{coucou}{{\begin{itemize}\item 1 \item 2 \end{itemize}}}{ça}{3 4}{va}{5 6}
\tripleitemiseavecTitre{Automatic control -- discrete and continuous time}%Automatique -- Temps continu ou échantillonné
{%
{\begin{itemize}[label=\myitem]%
\item \textbf{Modelling: }%Modélisation
 {\small State space, linear and non linear, linear multiple input-output, uncertain, time delays system}%fréquentielle, d'espace d'état, linéaire et non linéaire, linéaire multivariable, d'observateurs, incertaine, de système à retard.
% 
\item \textbf{Analysis: }%Analyse:
{\small Frequency, temporal (linear and non-linear), Lyapunov theory, performance, uncertain system, robustness, stability of times delays system}%fréquentielle, Théorie de Lyapunov, de performance, de système linéaire et non linéaire, de système incertain, de robustesse, de stabilité sur systèmes à retard.
\item \textbf{Control: }%Synthèse de commande:
{\small PID, multiple input-output, robust, Observer based state feedback, late system}%PID, multivariable, robuste, par retour d'état, sur système a retard.
\end{itemize}}}%
{Automatic control -- Discrete events systems}%Automatique -- Systèmes à événements discret :
{{\begin{itemize}[label=\myitem]%
		\item \textbf{Modelling : }%Modélsiation
		{\small Automaton, Petri net (standard, stochastic, timed), %Réseaux de Petri (normaux, stochastiques/temporisés),
		$(max,+)$ algebra, %Algèbre $(max,+)$,
		Discrete EVent Specification (DEVS), %Modèles DEVS
		Language}%Langage,
		\item \textbf{Analysis : }%Analyse
		{\small Cyclicity, controlability, diagnosability, determinism, coverage tree, marked and recognized language}
		\item \textbf{Control and diagnostic : }%Commande et diagnostic
		{\small Supervised control, diagnoser, observer}
		\item \textbf{Implementation : }%mise en oeuvre
		{\small Test, simulation, C, VHDL and ST implementations, Oriented object approach}
		\end{itemize}}}%
% {Automatic control -- Discrete events systems}%Automatique -- Systèmes à événements discret :
% {{\begin{itemize}[label=\myitem]%
% 		\item Automaton,%Automates
% 		\item {\small Petri network (standard, stochastic, timed),}%Réseaux de Petri (normaux, stochastiques/temporisés),
% 		\item {\small $(max,+)$ algebra,}%Algèbre $(max,+)$,
% 		\item {\small Discrete Event System Specification (DEVS) models,}%Modèles DEVS
% 		\item {\small control and controllability,}%Synthèse de commande et contrôlabilité,
% 		\item {\small Analysis, simulation, implementation,}%Analyse, simulation, implémentation,
% 		\item {\small Language,}%Langage,
% 		\item {\small Supervised control and diagnosers.}%Commande supervisée et diagnostiqueur,
% %		\item {\small Mise en \oe uvre logicielle/matérielle}%
% 		\end{itemize}}}%
%
{Implementation}%Mise en \oe uvre :
{{%
		\begin{itemize}[label=\myitem]%
\item \textbf{Computer science : }%Informatique : 
{\small System modelling (UML, UML2, SysML, embedded systems), object-oriented, parallel (mutual exclusion, synchronisation, thread, multitasking)}%Modélisation système (UML, UML2, SysML, systèmes embarqués), orienté objet, parallèle (exclusion mutuelle, synchronisation, thread , multitâches.).
\item \textbf{Industrial computing : }%Informatique industrielle : 
{\small DSP \textit{notions}, Microcontroller \textit{basics}, } %DSP \textit{notions}, Micro-contrôleurs \textit{bases},
\item \textbf{Real time : }%Temps Réel : 
{\small OSEK/VDX standard, scheduling, RTOS, requirement checking, reactivity}%Norme OSEK/VDX, Ordonnancement, RTOS, Vérification d’exigence, Réactivité.
\item \textbf{Network : }%Réseaux : 
{\small Internet \textit{basics}, Network Calculus, CAN, AFDX, real time network}%Internet \textit{bases}, Network Calculus, CAN, AFDX, Réseaux temps réel.
	\end{itemize}%
}}%
	
%	\item Langages : 
%		\begin{itemize}
%		\item 	\textbf{Matlab} \textit{bonnes connaissances}, 
%		\item \textbf{\LaTeX}  \textit{bonnes connaissances},
%		\item \textbf{C} \textit{bonnes connaissances}, 
%		\item \textit{Assembleur} \textit{notions},
%		\item \textbf{VHDL} \textit{bonnes bases}, 
%		\item \textbf{ST} et \textbf{IL-LIST} \textit{notions}, 
%		\item \textbf{Arduino} \textit{bonnes bases},
%		\item \textbf{Java} \textit{bases}, 		
%		\item \textbf{C++} \textit{notions}.
%		\end{itemize}
%		
%	\item Systèmes d'exploitation :
%		\begin{itemize}
%		\item \textbf{Linux} \textit{bonnes bases},
%		\item \textbf{Trampoline RTOS} \textit{bases}, 
%		\item \textbf{Windows} \textit{bonnes bases},
%		\item \textbf{Mac OS} \textit{bonnes bases}.
%		\end{itemize}
%	\end{itemize}		
%
%\end{itemize}
%}
\tripleitemiseavecTitre%
{Software skills}%Logiciels :
{{%
	For automatic : {\small \textbf{Matlab}: Simulink, OOP, GUI, RTW}\\ %Spécifiques automatique : {\small \textbf{Matlab}: Simulink, POO, GUI, RTW}.
	For computer science : {\small \textbf{Eclipse},\textbf{Git}, \textbf{Doxygen}}\\%Spécifique Informatique : {\small \textbf{Eclipse},\textbf{Git}, \textbf{Doxygen}}.
	Office software : {\small \textbf{\TeX maker}, \textbf{Microsoft office suite}, \textbf{Free Office Suite}}%Bureautique : {\small \textbf{\TeX maker}, \textbf{Suites Microsoft office}, \textbf{Libre Office}.
}}%
{Languages}%Langages :
{%
	\textbf{Matlab} \textit{good knowledge}, \textbf{\LaTeX}~\textit{good knowledge}, \textbf{C}~\textit{good knowledge}, 	\textbf{Assembler}~\textit{notion}, \textbf{VHDL} \textit{good foundation}, \textbf{ST}~and~\textbf{IL-LIST}~\textit{notion}, \textbf{Arduino}~\textit{good foundation}, \textbf{Java}~\textit{basics}, \textbf{C++}~\textit{notion}%
}%	\textbf{Matlab} \textit{bonnes connaissances}, \textbf{\LaTeX}~\textit{bonnes connaissances}, \textbf{C}~\textit{bonnes connaissances}, 	\textbf{Assembleur}~\textit{notions}, \textbf{VHDL} \textit{bonnes bases}, \textbf{ST}~et~\textbf{IL-LIST}~\textit{notions}, \textbf{Arduino}~\textit{bonnes bases}, \textbf{Java}~\textit{bases}, \textbf{C++}~\textit{notions}.%
{Language and communication skills}%Langue et communication
{{%
	\begin{itemize}[label=\myitem]%
	\item \textbf{Language: } {\small French (mother tongue),  English}%{Langue: } {\small Anglais B2}%
	\item \textbf{Communication: } {\small Oral and written in French and English} %{Communication: } {\small orale et écrite en Français et en Anglais.}
	\item \textbf{Project management: } {\small Gantt, WBS, RACI, Agile}%{Gestion et management de projet: } {\small Gantt, WBS, RACI, Agile.}
	\end{itemize}%
}}%
\vspace*{-4mm}% -----------------------
\partie{Personal interests}%Centres d'intérêts :
%\vspace*{-3mm}% -----------------------
%\tripleitemise[itemSymbole1]{text1}[itemSymbole2]{text2}[itemSymbole3]{text3}
\tripleitemise[\bcbook]{scientific watch}[\bcfleur]{Travels}[\bcvelo]{Do-it-yourself (bike trailer, electronic, …)}%{Veille scientifique}[\bcfleur]{Voyages}[\bcvelo]{Bricolage (remorque vélo, électronique, …)}
\end{samepage}%
\end{document}


%% end of file `template.tex'.
